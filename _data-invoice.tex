\RequirePackage{pgfcalendar}
\RequirePackage{pgfmath}

% ################## invoice DATA ##################
\newcommand{\rawInvoiceDate}{2025-07-27}
% ################## invoice DATA ##################

% ========= 1) Zählervariablen anlegen ====================
\newcount\invJul
\newcount\payJul
% ========= 2) Ausgangsdatum (ISO) → Julian ==============
\pgfcalendardatetojulian{\rawInvoiceDate}{\invJul}
% ========= 3) +14 Tage ohne pgfmath ======================
\payJul = \invJul
\advance\payJul by 14
% ========= 4) Formate erzeugen ===========================
\pgfcalendarjuliantodate{\invJul}{\invYear}{\invMonth}{\invDay}
\edef\invoiceDate{\invDay.\invMonth.\invYear}
\pgfcalendarjuliantodate{\payJul}{\payYear}{\payMonth}{\payDay}
\edef\payDate{\payDay.\payMonth.\payYear}

% ################## invoice DATA ##################
\edef\invoiceReference{RE-360-\invYear\invMonth\invDay-001}
% ################## invoice DATA ##################


% ################## invoice DATA ##################
\newcommand{\invoiceSalutation}{Sehr geehrte Damen und Herren,} % die Anrede
\newcommand{\invoiceText}{für die von mir erbrachte Leistung erhalten Sie
hiermit die Rechnung. Bitte zahlen Sie den unten aufgeführten Gesamtbetrag
unter Angabe der Rechnungsnummer (\invoiceReference) bis
zum \payDate \ auf das angegebene Konto ein.} % Rechnungstext
\newcommand{\invoiceEnclosures}{} % \encl{} einfügen
\newcommand{\invoiceClosing}{Mit freundlichen Grüßen}
% ################## invoice DATA ##################



%################### Invoice DATA ###################
\newcommand{\hourprice}{14.00}
\newcommand{\hournumber}{17}
%################### Invoice DATA ###################
